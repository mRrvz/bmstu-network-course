\section{Технологическая часть}

В данном разделе описываются средства разработки программного обеспечения и его сборка.

\subsection{Выбор средств разработки}

\subsubsection{Ядро Linux}

Разрабатываемый протокол был реализован на уровне ядра ОС Linux. Данный выбор был обусловлен несколькими факторами:

\begin{itemize}
	\item [---] Производительность: реализация шифрования на уровне ядра позволяет использовать аппаратные возможности процессора и обеспечивать высокую скорость обработки криптографических операций.
	\item [---] Универсальность: реализация на уровне ядра ОС позволяет использовать его в различных сетевых приложениях, независимо от их конкретной реализации.
	\item [---] Взаимодействие с другими слоями сетевой модели: более тесное взаимодействие протокола транспортного уровня с другими слоями сетевой модели OSI, такими как сетевой уровень (IP) и канальный уровень (Ethernet), что может улучшить общую эффективность и производительность сети.
	\item [---] Открытый исходный код ядра Linux.
\end{itemize}

\subsubsection{Выбор языка программирования}
Для реализации протокола был выбран язык программирования C с использованием стандарта C11 \cite{C11}. Данный выбор обоснован тем, что язык С является основным языком, на котором написано ядро Linux.

\subsection{Интерфейс сокетов}

Протокол был реализован с помощью интерфейса сокетов. В ядре Linux создание сокетов реализовано с использование системного вызова \texttt{socket}, который требует трех обязательных аргументов:

\begin{enumerate}
	\item семейство адресов сокета;
	\item тип сокетов;
	\item используемый протокол.
\end{enumerate} 

В качестве семейства адресов используется \texttt{AF\_INET}, тип сокетов -- \\ \texttt{SOCK\_DGRAM}. Таким образом, для того чтобы создать необходимый сокет с разработанным протоколом, нужно выполнить системный вызов \texttt{socket}. Пример вызова представлен в листинге \ref{code:socket}.

\begin{code}
	\captionof{listing}{Пример создания сокета}
	\label{code:socket}
	\inputminted
	[
	frame=single,
	framerule=0.5pt,
	framesep=20pt,
	fontsize=\small,
	tabsize=4,
	linenos,
	numbersep=5pt,
	xleftmargin=10pt,
	]
	{c}
	{code/socket.c}
\end{code}

\subsection{Сборка программного обеспечения}

\subsubsection{Прошивка устройства}

Сборка прошивки производится с помощью кросс-платформенной системы сборки buildroot \cite{buildroot}. В листинге \ref{code:defconfig_buildroot} представлена конфигурация, которая должна быть использована в процессе сборки прошивки устройства.

\begin{code}
	\captionof{listing}{Конфигурационный файл для сборки прошивки устройства}
	\label{code:defconfig_buildroot}
	\inputminted
	[
	frame=single,
	framerule=0.5pt,
	framesep=20pt,
	fontsize=\small,
	tabsize=4,
	linenos,
	numbersep=5pt,
	xleftmargin=10pt,
	]
	{text}
	{code/defconfig}
\end{code}

Прошивка подойдет для устройств Raspberry Pi 4 Model B.

\subsubsection{Сборка ядра Linux}

Ядро Linux собирается с помощью buildroot в виде пакета \texttt{linux}. Используемая версия ядра -- 6.6.

Сам протокол реализован в виде исходного файлов ядра, находящегося по адресу \texttt{net/ipv4/encrypted.c}. Это означает о том, что он собирается в составе Linux, а не как отдельный загружаемый модуль ядра. Для этого, был модифицирован один из Makefile ядра. Модификация представлена в листинге \ref{code:makefile}.

\begin{code}
	\captionof{listing}{Изменения в \texttt{net/ipv4/Makefile}}
	\label{code:makefile}
	\inputminted
	[
	frame=single,
	framerule=0.5pt,
	framesep=20pt,
	fontsize=\small,
	tabsize=4,
	linenos,
	numbersep=5pt,
	xleftmargin=10pt,
	]
	{text}
	{code/makefile}
\end{code}

\pagebreak